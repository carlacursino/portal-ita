\documentclass[10pt,twoside,report,glossaries,listings,nolistformulas,nolisttables,nolistalgorithms,final]{gt-report}

\usepackage[sfdefault,extralight]{FiraSans}

%
% Definir como "digital" ou "printed" para ter a versão para ser distribuída digitalmente ou impressa
% Cada pesquisador define o conteúdo a ser impresso ou não com blocos:
%
% \begin{shownto}{digital}
% . . .
% \end{shownto}{digital}
%
\DefCurrentAudience{printed}

\graphicspath{
    {../_common/},
    {../RS0001/anexos/},
    {../RS0006/anexos/},
}

\pagestyle{fancy}

\addbibresource{../_common/Referencias/bibliografia.bib}

\addbibresource{../RS0001/_bibliography.bib}
\addbibresource{../RS0002/_bibliography.bib}
\addbibresource{../RS0006/_bibliography.bib}

\loadglsentries{../_common/Referencias/glossario}

\begin{titlepage}
    \title{\textbf{Portal Institucional do ITA \\ \normalsize{Manual do Editor}}}
    \institution{ITA - Instituto Tecnológico de Aeronáutica}
    \department{IEC - Divisão de Ciência da Computação}
    \author{Carla Cursino\orcidIcon{0000-0002-7718-5897}\\
    %Prof. Dr. Luiz Alberto Vieira Dias\orcidIcon{0000-0002-6544-7458}\\
}

\end{titlepage}

\begin{document}

\usCopy{RS0006}{Guia do Editor}{..}{}
\usCopy{RS0002}{Guia de Cores}{..}{}
\usCopy{RS0001}{Guia de Estilos}{..}{}

% \clearpage

% \begin{appendices}\label{sec:appendix}
%     \renewcommand\thechapter{\Alph{chapter}}
%     \pretocmd{\chapter}{%
%         \clearpage
%         \pagenumbering{arabic}%
%         \renewcommand*{\thepage}{\thechapter\arabic{page}}%
%     }{}{}

%     \paragraph{Observações sobre arquivos anexos}

%     \begin{displayquote}\footnotesize
%         \begin{enumerate}
%             \item Anexos de arquivo são um recurso da especificação PDF 1.3. Eles não serão mostrados em visualizadores de PDF que não suportam a especificação PDF 1.3. (A versão 4.0 do Adobe Acrobat é a primeira versão desse programa que a suporta.);
%             \item Mesmo algumas instalações que supostamente suportam a especificação PDF 1.3 não suportam anexos de arquivo. Pelo que sabemos, versões muito antigas do Adobe Acrobat Reader (a versão gratuita e de visualização do Adobe Acrobat) não parecem suportar nenhuma anotação, exceto as de texto;
%             \item Mesmo alguns visualizadores que suportam especificação PDF 1.3 e anexos de arquivos não os suportam em todas as circunstâncias. Por exemplo, algumas versões do Adobe Acrobat para Windows, quando funcionam como um \textit{plugin} de navegador da Web, apresentam uma mensagem de erro quando um ícone de anexo de arquivo é ativado;
%             \item Mesmo nas circunstâncias em que os anexos de arquivos são suportados, o suporte pode apresentar falhas. Por exemplo, algumas versões do Adobe Acrobat para Windows alteram um ícone personalizado para o ícone padrão quando ele é selecionado;
%             \item Embora os ícones de anexo de arquivo com aparência personalizada tenham sido impressos com boa qualidade nas versões mais antigas do Adobe Acrobat, a Adobe introduziu uma falha a partir do Adobe Acrobat 6.0 que impede a impressão dos ícones do anexo. Infelizmente, como o Adobe Acrobat não possui a capacidade, de no arquivo, habilitar aparências personalizadas para ícones de anexos, é improvável que essa falha seja corrigida.
%             %\item Caso não consiga obter um dos anexos deste relatório, solicite sua cópia enviando e-mail para o autor \url{gomesjm@ita.br}.
%         \end{enumerate}
%     \end{displayquote}\normalsize

%     \clearpage

%     \usAnnex{RS0001}{}{..}
%     \usAnnex{RS0002}{}{..}
%     \usAnnex{RS0006}{}{..}

% \end{appendices}

\end{document}