%% usamos: Termo1-Termo2-{...}-TermonN      para siglas fechadas, ou seja, valores ou 
%%                                          passos de um método onde todos são necessários
%%         Termo1, Termo2, {...}, {TermoN}  para siglas aberts, ou seja, valores ou
%%                                          passos de um método onde alguns são opcionais
%%         Termo1 Termo2 {...} TermoN       para nomes de pessoas ou lugares
%%
%% a sigla é repetida em negrito, seguido da tradução em português (em um dos formatos acima), 
%% seguida da tradução em inglês italizada

\newacronym{LDAP}{LDAP}{Lightweight Directory Access Protocol}

\newacronym{LDIF}{LDIF}{\gls{LDAP} Data Interchange Format}

\newacronym{ITA}{ITA}{Instituto Tecnológico de Aeronáutica}

\newglossaryentry{browser}{name={browser},description={Aplicativo utilizado para apresentar conteúdo da internet num computador pessoal}}

\newacronym{URL}{URL}{Uniform Resource Locator, ou endereço internet, é uma referência completa para um recurso localizado num servidor na internet, podendo ser um documento, imagem ou mídias tais como som ou vídeo}

\newacronym{URI}{URI}{Uniform Resource Identifier, ou indentificador de recurso num servidor na internet, é uma referência para o recurso, podendo ser um documento, imagem ou mídias tais como som ou vídeo}

\newacronym{CMS}{CMS}{Content Management System, um sistema para gestão de conteúdo para portais na internet}

\newacronym{HTML}{HTML}{Linguagem de formatação de documentos, do inglês ``\textit{Hypertext Markup Language é, do inglês}'' (Linguagem de Marcação Hipertexto)}

\newacronym{WYSIWYG}{WYSIWYG}{Em computação WYSIWYG é, do inglês, ``\textit{What You See Is What You Get}'' (O que você vê é o que você obterá)}