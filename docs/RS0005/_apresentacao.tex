\makeatletter
    \def\input@path{{../_common/}}
\makeatother

\documentclass[usenames,dvipsnames]{gt-beamer}

%
% begin MINTED
%
%   ATENÇÃO
%
%   Esta classe está usando o pacote "minted", que por usa vez depende de
%   Python 2.6+ e da biblioteca "pygmentize" do Python
%
%   No MacOS basta usar o comando abaixo para instalar o pygmentize:
%
%       sudo easy_install pygments
%
%   Além disto, o pdflatex precisa do parâmetro "-shell-escape" para invocar o
%   "pygments".
%
%   O arquivo "Makefile" já inclui o parâmetro, mas caso utilize alguma IDE
%   esteja atento para este fato - e não habilite o parâmetro "-shell-escape"
%   de modo global, mas para apenas documentos específicos - isto pode representar uma
%   falha de segurança enorme!
%
% \usepackage[newfloat]{minted}
%
% end MINTED
%

%% usamos: Termo1-Termo2-{...}-TermonN      para siglas fechadas, ou seja, valores ou 
%%                                          passos de um método onde todos são necessários
%%         Termo1, Termo2, {...}, {TermoN}  para siglas aberts, ou seja, valores ou
%%                                          passos de um método onde alguns são opcionais
%%         Termo1 Termo2 {...} TermoN       para nomes de pessoas ou lugares
%%
%% a sigla é repetida em negrito, seguido da tradução em português (em um dos formatos acima), 
%% seguida da tradução em inglês italizada

\newacronym{LDAP}{LDAP}{Lightweight Directory Access Protocol}

\newacronym{LDIF}{LDIF}{\gls{LDAP} Data Interchange Format}

\newacronym{ITA}{ITA}{Instituto Tecnológico de Aeronáutica}

\newglossaryentry{browser}{name={browser},description={Aplicativo utilizado para apresentar conteúdo da internet num computador pessoal}}

\newacronym{URL}{URL}{Uniform Resource Locator, ou endereço internet, é uma referência completa para um recurso localizado num servidor na internet, podendo ser um documento, imagem ou mídias tais como som ou vídeo}

\newacronym{URI}{URI}{Uniform Resource Identifier, ou indentificador de recurso num servidor na internet, é uma referência para o recurso, podendo ser um documento, imagem ou mídias tais como som ou vídeo}

\newacronym{CMS}{CMS}{Content Management System, um sistema para gestão de conteúdo para portais na internet}

\newacronym{HTML}{HTML}{Linguagem de formatação de documentos, do inglês ``\textit{Hypertext Markup Language é, do inglês}'' (Linguagem de Marcação Hipertexto)}

\newacronym{WYSIWYG}{WYSIWYG}{Em computação WYSIWYG é, do inglês, ``\textit{What You See Is What You Get}'' (O que você vê é o que você obterá)}

% \setbeameroption{show notes}    % Comente esta linha para não imprimir as anotações

%
% Definir como "digital" ou "printed" para ter a versão para ser distribuída digitalmente ou impressa
% Cada pesquisador define o conteúdo a ser impresso ou não com blocos:
%
% \begin{shownto}{digital}
% . . .
% \end{shownto}
%
% \SetNewAudience{printed}
% \SetNewAudience{digital}
% \DefCurrentAudience{printed}

\graphicspath{{./anexos/}}
\DeclareGraphicsExtensions{.pdf,.png,.jpg}

\title{Geração Automática de Testes}

\subtitle{Testes Generativos}

\institute[ITA]
{
  \inst{1}Instituto Tecnológico de Aeronáutica - ITA
}

\author[ADA] {
    Pesq. \textbf{Lovelace}, A.\inst{1}\\
    Prof. Dr. \textbf{Torvalds}, L.\inst{1}
}

\us{9999}

\usSprint{33}{Fev-2020}

\date[\usDate]{Projeto STAMPS,~Sprint~\usSprint,~US~\usNumber}

\subject{Testes de Software, Testes Generativos, Automação de Testes de Software, Geração Automática de Testes}

\begin{document}

%
% Trecho de código que será inserido num bloco colorizado pelo minted
%
%     \defverbatim[colored]\kleeExemplo{
%         \begin{minted}[breaklines]{c}
% int main() {
% int x, y;
%
% klee_make_symbolic(&x, sizeof(x), "x");
% klee_make_symbolic(&y, sizeof(y), "y");
% int *p = f(x, y);
% return 0;
% }
%         \end{minted}
%     }
%
% . . .
%
% Depois no bloco:
%
% \begin{block}{Execução simbólica - função}
%     \kleeExemplo
% \end{block}
%

    % \usTOC                % - Apresenta frames intermediários com a agenda da apresentação

    \usCover{ita_logo}

    \section{A User Story}

    \begin{frame}[allowframebreaks]
    	\frametitle{Sprint \usSprint, \usDate}

        \usStory
            {aplicar uma template à apresentação}
            {facilitar a vida do pesquisador}

        \begin{block}{Critérios de Aceitação}
            \begin{enumerate}
                \item
                    \usCriteria
                        {uma template}
                        {for bonita e funcional}
                        {usar para fazer a apresentação}
                \item
                    \usCriteria
                        {outra template}
                        {for feia e disfuncional}
                        {descartar}
            \end{enumerate}
        \end{block}

    \end{frame}

    \section{Introdução}

    \begin{frame}[allowframebreaks]
        \frametitle{Cenário}

        \begin{block}{Caos na \textit{Sprint Review}}
            \begin{enumerate}
                \item Reportado erro crítico na template;
                \item Encerrar a questão.
            \end{enumerate}
        \end{block}

    \end{frame}

    \note{
        \begin{itemize}
            \item Isto é uma nota de apresentação que somente será mostrada se aquele primeiro comentário lá em cima for ``descomentado'';
            \item Estamos fazendo na forma de ítens para facilitar a organização e leitura;
            \item O apresentador gera uma versão com as notas para si próprio;
            \item E outra versão é gerada para apresentar no telão (sem as notas).
        \end{itemize}
    }
    \note{
        \begin{itemize}
            \item Outro bloco de notas - será apresentado numa nova página de notas;
            \item A numeração de páginas da apresentação não muda!
        \end{itemize}
    }

    \begin{frame}[allowframebreaks]
        \frametitle{Os custos de uma péssima apresentação}

        \begin{block}{Custos com erros que repercutem durante a apresentação}
            \begin{enumerate}
                \item Cochilos;
                \item Broncas do coordenador do projeto;
                \item Desinteresse dos ``stake-holders'' em prosseguir com a pesquisa.
            \end{enumerate}
        \end{block}
    \end{frame}

    \addtocontents{toc}{\newpage}

    \section{Evolução}

    \begin{frame}[allowframebreaks]
        \frametitle{Organização do trabalho}

        \begin{block}{Aonde queremos chegar}
            \begin{enumerate}
                \item \color{WildStrawberry}Apresentar;
                \item \color{OliveGreen}De forma;
                \item \color{MidnightBlue}Organizada.
            \end{enumerate}
        \end{block}

    \end{frame}

    \note{
        \begin{itemize}
            \item Para realizar este trabalho, iniciamos por um diagnóstico da implementação de apresentações;
            \item Fizemos um beamer;
            \item Que depois foi convertido na presente classe.
        \end{itemize}
    }

    \addtocontents{toc}{\newpage}

    \section{Resultados}

    \begin{frame}[allowframebreaks]
        \frametitle{Notas sobre os resultados e anexos}

        \begin{block}{Números}
            \begin{enumerate}
                \item $\sim 40.000$ linhas de código
                \item $\sim 1.240$ funções públicas
                \item $\sim 100$ testes ``integrados''
                \item $0$ testes ``unitários''
            \end{enumerate}
        \end{block}

    \end{frame}

    \section{Conclusão}

    \begin{frame}[allowframebreaks]
        \frametitle{Conclusão}

        \usStory
            {aplicar uma template à apresentação}
            {facilitar a vida do pesquisador}

        \begin{block}{Critérios de Aceitação}
            \begin{enumerate}
                \item
                    \usCriteria
                        {uma template}
                        {for bonita e funcional}
                        {usar para fazer a apresentação}
                \item
                    \usCriteria
                        {outra template}
                        {for feia e disfuncional}
                        {descartar}
            \end{enumerate}
        \end{block}

        \begin{block}{Checklist}
            \begin{itemize}
                \item Critérios de Aceitação
                    \begin{itemize}
                        \item Template Bonita - \color{green}\ding{52}\color{black}
                        \item Template Feia - \color{red}\ding{56}\color{black}
                    \end{itemize}
                \item Objetivos - \color{green}\ding{52}\color{black}
            \end{itemize}
        \end{block}

        \framebreak

        \begin{block}{Genéricas}
            \begin{itemize}
                \item Soluções atendem qualquer tipo de apresentação;
                \item Podem ser adotadas por qualquer tamanho de equipe.
            \end{itemize}
        \end{block}

        \note{
            \begin{itemize}
                \item As soluções adotadas neste trabalho atendem a qualquer tipo de projeto de pesquisa e foram baseados em nossa experiência em projetos de naturezas, portes, plataformas e arquiteturas variadas e podem ser aplicados tanto por equipes de um homem só quanto por grandes times multidisciplinares.
            \end{itemize}
        }

        \framebreak

        \begin{block}{Específicas}
            \begin{itemize}
                \item Soluções apresentadas não se limitam somente à pesquisas;
                \item Teses podem ser apresentadas utilizando-se esta template.
            \end{itemize}
        \end{block}

        \note{
            \begin{itemize}
                \item Algumas das recomendações e sugestões apresentadas pelo presente relatório e resumidas a seguir não se limitam somente à apresentações.
            \end{itemize}
        }

    \end{frame}

    \note{
        \begin{itemize}
            \item Assim, partindo-se do cenário apresentado e dos critérios de aceitação propostos, iniciamos com uma apresentação.
            \item A adoção das técnicas apresentadas e resumidas na seção de recomendações deste relatório garantem a mitigação dos problemas encontrados com a baixa padronização das apresentações.
        \end{itemize}
    }

    \begin{frame}[allowframebreaks]
        \frametitle{Sugestões \& Recomendações}

        \begin{block}{Recomendações}
            \begin{itemize}
                \item Implementar algumas coisas por meio de \color{blue}cor azul\color{black};
                \item Outras devem ser \color{orange}laranja\color{black}.
            \end{itemize}
        \end{block}

        \framebreak

        \begin{block}{Sugestões}
            \begin{itemize}
                \item Separação de cores e contrastes.
            \end{itemize}
        \end{block}

        \note{
            \begin{itemize}
                \item Seria uma solução ótima do ponto de vista de separação de preocupações e níveis de abstração.
            \end{itemize}
        }

    \end{frame}

    \usFinish{XKCD_is_it_worth_the_time}

\end{document}