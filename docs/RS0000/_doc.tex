\section{Arquitetura}\label{RS0000:arquitetura}

\subsection{Decisões Arquiteturais}

Todas as necessidades para implementação do Portal e o Painel Administrativo passaram por um processo decisório onde aplicou-se o recurso de registrar um cartão chamado de Registro de Decisão Arquitetural\cite{kopp2018markdown}.

O formato adotado foi o seguinte:

\begin{code}
    \inputminted[label=rda]{text}{../RS0000/anexos/exemplo.rda}
    \caption{Exemplo de Registro de Decisão Arquitetural}\label{RS0000:code:exemplo-rda}
\end{code}

\subsubsection{Armazenamento de Dados}

\begin{code}
    \inputminted[label=Armazenamento de Dados]{text}{../RS0000/anexos/db.rda}
    \caption{RDA Banco de Dados}\label{RS0000:code:db-rda}
\end{code}

Com base nos critérios acima escolhemos o gerenciador de banco de dados \textit{NoSQL} \textbf{MongoDB} que com base em nossa experiência anterior atende a todos listados.

\subsubsection{Gerenciamento de Conteúdo}

\begin{code}
    \inputminted[label=Gerenciamento de Conteúdo]{text}{../RS0000/anexos/framework.rda}
    \caption{RDA Framework CMS}\label{RS0000:code:framework-rda}
\end{code}

Com base nos critérios acima foram considerados dois \textit{frameworks}:

\begin{enumerate}
    \item Wagtail - Linguagem \textbf{Python} - completude e complexidade alta
    \item KapstoneJS - Linguagem \textbf{JavaScript} - completude baixa e complexidade baixa
\end{enumerate}

Escolhemos o \textit{framework} \textbf{KeystoneJS}, que algum tempo depois de iniciado o projeto evoluiu para uma nova versão incompatível com a que estávamos utilizando e sem o provisionamento de um plano de migração. Como um dos critérios foi justamente o de que o \textit{framework} escolhido deveria ser escrito numa linguagem reconhecida pelo mercado e de código aberto e com o investimento em tempo e aprendizado já feito na ferramenta, decimos pela \textit{derivação} do projeto original e assim nasceu o \textbf{CapstoneJS}.

Seguindo a filosofia proposta, todas as funcionalides pedidas pelos usuários da instituição foram implementadas utilizando recursos do projeto original, e para atendê-las foram criados sub-projetos, alguns derivados de projetos originais voltados ao \textbf{KeystoneJS}, outros adaptados para funcionarem tanto neste quanto em nossa derivação, e que listamos:

\begin{enumerate}
    \item CapstoneJS - Gerenciador de Conteúdo
    \item capstone-file-manager - Gerenciador de arquivos dos usuários
    \item capstone-tinymce - Integração do editor de textos TinyMCE ao CapstoneJS
    \item capstone-intl - Suporte multi-linguagem
    \item capstone-accesscontrol - Controle de acesso à modificação de conteúdo por meio de regras
    \item capstone-auth - Authenticação com serviço LDAP ou Active Directory
\end{enumerate}
